% Packages for images
\usepackage{float}
\usepackage{svg}
\usepackage{graphicx}
\usepackage{caption}
\usepackage{subcaption}
\usepackage{capt-of}

% Packages for diagrams
\usepackage{pgfplots}
\pgfplotsset{compat=1.18}
\usepgfplotslibrary{fillbetween}
\pgfmathdeclarefunction{gauss}{2}{%
	\pgfmathparse{1/(#2*sqrt(2*pi))*exp(-((x-#1)^2)/(2*#2^2))}%
}

\usepackage{tikz}
\usetikzlibrary{patterns}
\usetikzlibrary{decorations.pathmorphing}
\usetikzlibrary{hobby}
\usetikzlibrary{shapes.arrows}

\usetikzlibrary{decorations.markings}
\tikzset{
	set arrow inside/.code={\pgfqkeys{/tikz/arrow inside}{#1}},
	set arrow inside={end/.initial=>, opt/.initial=},
	/pgf/decoration/Mark/.style={
			mark/.expanded=at position #1 with
				{
					\noexpand\arrow[\pgfkeysvalueof{/tikz/arrow inside/opt}]{\pgfkeysvalueof{/tikz/arrow inside/end}}
				}
		},
	arrow inside/.style 2 args={
			set arrow inside={#1},
			postaction={
					decorate,decoration={
							markings,Mark/.list={#2}
						}
				}
		},
}

\usepackage[compat=1.1.0]{tikz-feynman}


% Package for tables
\usepackage{booktabs}
\usepackage{longtable}
\usepackage{multirow}
\usepackage[l3]{csvsimple}      % read csv files into latex
